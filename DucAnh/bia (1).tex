\documentclass{article} % Tạo một bản báo cáo
\usepackage[utf8]{inputenc} 
\usepackage[T5]{fontenc} % Để sử dụng Tiếng Việt 
\usepackage[fontsize=13pt]{scrextend} % Set fontsize=13pt
\usepackage[paperheight=29.7cm,paperwidth=21cm,right=2cm,left=3cm,top=2cm,bottom=2.5cm]{geometry} % Chuẩn A4, căn lề phải, trái, trên, dưới.
\usepackage{mathptmx} % Time New Roman
\usepackage{graphicx} % Thư viện chèn ảnh
\usepackage{float} % Set vị trì chèn ảnh 
\usepackage{tikz} % Thư viện tạo khung bìa 
\usetikzlibrary{calc} % Thư viện tikz
\begin{document}
\begin{titlepage}
\begin{tikzpicture}[overlay,remember picture]
\draw [line width=3pt]
    ($ (current page.north west) + (3.0cm,-2.0cm) $)
    rectangle
    ($ (current page.south east) + (-2.0cm,2.5cm) $);
\draw [line width=0.5pt]
    ($ (current page.north west) + (3.1cm,-2.1cm) $)
    rectangle
    ($ (current page.south east) + (-2.1cm,2.6cm) $);
\end{tikzpicture}
\begin{center}
\vspace{-15pt} TRƯỜNG ĐẠI HỌC SÀI GÒN\\
\textbf{\fontsize{16pt}{0pt}\selectfont KHOA TOÁN ỨNG DỤNG}
\vspace{0.5cm}
\begin{figure}[H]
    \centering
    \includegraphics[width=6cm,height=6cm]{bianckh.png}
\end{figure}
\vspace{1.5cm}
\fontsize{24pt}{0pt}\selectfont BÁO CÁO\\
\vspace{12pt}
\textbf{\fontsize{32pt}{0pt}\selectfont NGHIÊN CỨU KHOA HỌC}
\vspace{1.5cm}
\end{center}
\end{titlepage}
\end{document}