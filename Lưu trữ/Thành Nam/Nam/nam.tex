\documentclass{article}
\usepackage{graphicx} % Required for inserting images
\usepackage[utf8]{vietnam}
\usepackage{float}
\usepackage{graphicx}
\usepackage{tikz}
\usepackage{amssymb}
\usepackage{amsmath}
\usepackage{amsfonts}
\usepackage{cases}
\usepackage{braket}
\usetikzlibrary{calc}
\usepackage[left=2cm,right=2cm,top=2cm,bottom=2cm]{geometry}
\usepackage{multicol}
\usepackage{enumerate} %thư viện liệt kê,đánh số có cấp độ
\usepackage{fontawesome}% thư viện kí tự đặc biệt của Latex
\usepackage{xcolor} % Sử dụng gói xcolor
\usepackage{colortbl} % Sử dụng gói colortbl
\usepackage{hyperref}
\usepackage{blindtext}
\usepackage{titlesec}
\usepackage{array}
\usepackage{mathtools}
\usepackage{tabularx}
\usepackage{geometry}
\usepackage{nccmath}
\usepackage{ulem}


\begin{document}
\Large\section*{Một số Code thông dụng trong toán học}
\large $\setminus$quad là khoảng cách trắng\\

\LaTeX\

$\setminus$ qquad là double khoảng cách trắng,chấm phẩy và phẩy cũng là khoảng cách trắng\\

$\setminus$allowdisplaybreaks là cho phép ngắt công thức toán nếu đã dài quá trang sang trang mới\\

\colorbox{pink}{$\setminus$geqslant} là lớn hơn bằng, \colorbox{pink}{$\setminus$leqslant}là nhỏ hơn bằng.\\

\colorbox{yellow}{$\setminus$geq} là lớn hơn,\colorbox{yellow}{$\setminus$leq}là nhỏ hơn

\textcolor{blue}{aligned} làm hpt gần nhau hơn và giữa các pt ko có khoảng cách quá lớn(đẹp,gọn)\\

\textcolor{blue}{cases} làm cho hpt xa nhau và kcach giữa các pt lớn (ko đẹp)\\

Hết một lệnh $\setminus$draw thì phải chấm phẩy.

$\setminus$colorbox\{màu\}\{văn bản\} là tô màu nền bạn thích cho phần \{văn bản\}.\\

\colorbox{yellow}{ctrl double click} trên dòng nào từ file pdf sẽ nhảy tới dòng code chỗ dòng văn bản pdf đó.\\

\colorbox{pink}{1}$\quad x^2=0$  \quad     mũ bình thường \\

\colorbox{pink}{2}$\quad x^{100}$ \quad ,mũ lớn bỏ trong ngoặc nhọn\\

\colorbox{green}{3}$\quad x_{11}$ \quad chỉ số dưới bỏ ngoặc nhọn\\

\colorbox{red}{4}$\quad  x-2=0 \Rightarrow x=2 $\quad  $\setminus$Rigtharrow viết hoa là mũi tên to\\

\colorbox{red}{5}$\quad x-2=0 \rightarrow x=2  $\quad  ko viết hoa là mũi tên nhỏ\\

\colorbox{red}{6}$\quad x-2=0 \Leftrightarrow x=2 $\quad mũi tên hai chiều\\

\colorbox{red}{7}$\quad\dfrac{2x^2-5}{x^5-3}$\quad frac là chia,dfrac là phóng to phân số\\

\colorbox{red}{8}$\quad \displaystyle\int_a^bf(x)$ \quad ko kéo cận dãn ra,thêm limits là kéo cận dãn ra\\

\colorbox{red}{9}$\quad\displaystyle\int\limits_a^bf(x)$ (với limits là kéo cận lên trên và dưới của dấu tích phân,display là phóng to,int là tích phân)\\

\colorbox{red}{10}$\quad\displaystyle\sum\limits_a^b$\quad sum là tổng,limits là chỉ số trên dưới như tích phân\\

\colorbox{red}{11}$\quad\left(\dfrac{1}{2}\right)$ \quad lệnh left(...right) là giúp cho dấu ngoặc (tròn) bao hết phân số,tương tự với ngoặc vuông,ngoặc nhọn.\\

\colorbox{red}{12}$\quad\cos x$\quad hàm lượng giác phải thêm dấu khai báo\ vào trước tên của hàm đó\\

 \colorbox{red}{13}$\quad\begin{array}{cr}
 x-2=0 & 2x-5=0\\
 x-1=0 & x=0\\
 \end{array}$\\
 array là tạo bảng nhưng ko có viền, c là căn giữa,r là căn phải, cr theo thứ tự là cột 1 cột 2, dấu 'và' ngăn cách 2 cột c và r\\
 \colorbox{red}{14}\begin{align*}
 x-2=0\\
 x^2=4
 \end{align*}\\
 align* căn giữa công thức toán .ko đánh số pt ct toán\\
\colorbox{red}{15} $\quad\begin{aligned}
 &x-2=0\\
 &x^2=4
 \end{aligned}$\\
 aligned là căn trái công thức toán, dấu'và' đứng trc vị trí nào thì sẽ căn hàng tiếp theo đúng tại vị trí đó\\
 
 \colorbox{red}{16} Ta có hệ phương trình: $\left\{\begin{aligned}
     & 2x-3y=0\\
     & 5x-8y=4
 \end{aligned}\right.$\\
 cách viết hệ pt hoặc hệ hoặc,nếu ngoặc nhọn thì thêm dấu $\setminus$ vào sau chữ left,còn ngoặc [ ] thì bỏ dấu $\setminus$ đi.\\

\colorbox{red}{17} Đánh số liệt kê
 \begin{enumerate}
 \item Nội dung ý thứ nhất.
 \begin{enumerate}[i.]
 \item nội dung nhỏ ý 1
 \item nội dung nhỏ ý 1
 \end{enumerate}
 \item Nội dung ý thứ hai.
 \end{enumerate}

\colorbox{red}{18}  Dấu [ ] đánh số [bước 1.] luôn để tạo thành 1 bài toán có loạt câu hỏi
\begin{enumerate}[Bước 1.]
\item Tìm nguyên hàm sau $\displaystyle\int xdx $
\item Tìm giá trị tích phân dựa trên kq Bc1 $\displaystyle\int\limits_1^2f(x)dx$
\end{enumerate}\\

\colorbox{red}{19}\quad Hai bài toán trên được viết lại ngắn gọn dưới dạng ma trận như sau:
\begin{align*}
       f(x) =\langle c.x \rangle&= c^Tx \longrightarrow \text{Max}   & g(y) =\langle b,y \rangle&= b^Ty \longrightarrow \text{Min}\\
      Ax &\leq b & A^Ty &\geq c \\
     x &\geq 0 &  y &\geq 0
\end{align*}\\

\colorbox{green}{20}\\

             \begin{tabular}{|c|c|c|c|c|}
                \hline %dòng 1
                \rowcolor{pink} 1 & 2 & 3 & 4 & 5 \\ % Tô màu xanh lá cây dòng 1
                \hline % Dòng 2
                \cellcolor{yellow} G & F & D & 7 & 8 \\ % Tô cột đầu tiên dòng 2 màu vàng
                \hline
                K & L & \cellcolor{red} U & I & O \\ % Tô cột 3 dòng 3 màu đỏ ( chỉ 1 ô) 


             \end{tabular}\\




\colorbox{brown}{21} \\
% tô màu công thức toán học
\begin{center}
\color{red} $ x_1= \dfrac{-b+  \sqrt\triangle}{2a}$;\quad
\color{red} $x_2 = \dfrac{-b-\sqrt\triangle}{2a}$
\end{center}\\

\colorbox{red}{22}\\

\textcolor{green}{Shift Alt mũi tên xuống} là Duplicate dòng đó xuống thêm 1 dòng\\
\textcolor{green}{Ctrl C} ko càn bôi đen cae dòng đó,chỉ cần click chuột vào dòng đó là xong\\
 
\colorbox{red}{23}\\ % \[gõ văn bản toán học vào đây \] là căn giữa,ko cần sài $$...$$
\[\sqrt{2}\] % Hai dấu đồng tiền liền kề là căn giữa công thức toán và nằm riêng biệt trên một hàng
\[\sqrt[3]{2}\] % [] là căn bậc, {} là số hay biểu thức dưới căn bậc
\[\dfrac{1+\sqrt{1+x}}{\sqrt{x+2}}\]\\

\colorbox{red}{24}\\

\[\left\{\begin{aligned} 
    & x_j^* > 0 \Longrightarrow \displaystyle\sum\limits_{i=1}^na_{ij}y_i^*=c_j \;\vee\;\displaystyle\sum\limits_{i=1}^na_{ij}y_i^* < c_j\Longrightarrow x_j^* =0,j=\overline{1,n}\\
    & y_j^* > 0 \Longrightarrow \displaystyle\sum\limits_{i=1}^na_{ij}x_i^*=b_j \;\vee\;\displaystyle\sum\limits_{i=1}^na_{ij}x_i^* < b_j\Longrightarrow y_j^* =0,j=\overline{1,m}
\end{aligned}\right.\]

\colorbox{red}{25}\\

$\setminus$\textcolor{cyan}{setcounter}\{tocdepth\}\{1\} nghĩa là đánh số liệt kê trong mục lục cấp độ 1\\
VD: Mục Lục\\
\begin{enumerate}
    \item Mục 1
    \begin{enumerate}[1.]
    \item mục nhỏ 1  của mục 1 setcounter tocdepth \{1\}
    \item mục nhỏ 2  của mục 1 setcounter tocdepth \{1\}
    \end{enumerate}
    \item Mục 2
\end{enumerate}.\\
cứ thể mà từ cấp 1 đến cấp 5\\


 \colorbox{red}{26}\\

\begin{numcases}{}
    x^2=2 &\\
    x^4=2 &\\
    ... &\\
    x-2=0 &\\
\end{numcases}\\
\colorbox{red}{27}\\
\textbf{1.Cho bài toán quy hoạch tuyến tính như sau:}\\

$\left(P\right)$ $f(x)=2x_1-3x_2+4x_3-6x_4\quad \longrightarrow Min$\\

\[\left\{\begin{aligned}
    x_1+2x-2+3x_3-x_4&=20 \quad(1)\\
    -3x_1-x_2+7x_3+7x_4&\leq 32 \quad(2)\\
    2x_1+4x_2+x_3+x_4&\geq 18 \quad(3)\\
    x_i\geq 0,\forall i=1,2,3
\end{aligned}\right.\]\\
$\Longrightarrow$Ta có bài toán đối ngẫu như sau :$ \big(D\big)\quad g(y)=20y_1+32y_2+18y_3 \longrightarrow Max$ \\
\[\begin{cases} y_1-3y_2+2y_3&\leq 2 \quad(4)\\
    2y_1-y_2+4y_3&\leq-3 \quad(5)\\
    3y_1+7y_2+y_3&\leq 4 \quad(6)\\
    -y_1+7y_2+y_3&=-6 \quad(7)\\
    y_2\leq 0,y_3\geq 0
\end{cases}\]\\
Cặp ràng buộc : $\begin{array}{cr} 
    x_1\geq 0\; \&(4) &y_2\leq 0\; \&(2)\\
    x_2\geq 0\; \&(5) &y_3\leq 0\; \&(3)\\
\end{array}$


\colorbox{red}{28}\\
\[\begin{cases}a_{11}y_1+a_{21}y_2 &\geq c_1 \\
a_{12}y_1+a_{22}y_2 &\geq c_2 \\
a_{13}y_1+a_{23}y_2 &\geq c_3 \\
a_{14}y_1+a_{24}y_2 &\geq c_4 \\
y_i \geq 0,i=1,2,3,4
\end{cases}\].\\


\newpage


\colorbox{red}{29}
$\displaystyle\sum\limits_a^b(n+1)$\\

\colorbox{red}{30}\\
$\setminus$hfill$\setminus$par dùng trong định lí\\
dùng để xuống dòng và thụt đầu dòng.


    




\end{document}
