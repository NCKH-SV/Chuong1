\documentclass{article}
\usepackage[utf8]{vietnam} 
\usepackage[vietnamese]{babel}
\usepackage[T5]{fontenc} % Để sử dụng Tiếng Việt 
\usepackage[fontsize=13pt]{scrextend} % Set fontsize=13pt
\usepackage[paperheight=29.7cm,paperwidth=21cm,right=2cm,left=3cm,top=2cm,bottom=2.5cm]{geometry} % Chuẩn A4, căn lề phải, trái, trên, dưới.
\usepackage{mathptmx} % Time New Roman
\usepackage{graphicx} % Thư viện chèn ảnh
\usepackage{hyperref}
\usepackage{amsfonts,amsmath,amssymb} % Khai báo công thức toán học
\usepackage{float} % Set vị trì chèn ảnh 
\usepackage{tikz} % Thư viện tạo khung bìa 
\usepackage{multicol}
\usepackage{sectsty} %thư viện căn giữa lề section
\usepackage{enumerate} %thư viện liệt kê,đánh số có cấp độ
\usepackage{fontawesome}% thư viện kí tự đặc biệt của Latex
\usepackage{xcolor} % Sử dụng gói xcolor
\usepackage{colortbl} % Sử dụng gói colortbl
\usepackage{multicol}
\usetikzlibrary{calc} % Thư viện tikz
\usepackage{indentfirst} % Thư viện thụt đầu dòng 
\usepackage{ulem} % lệnh gạch dưới
\sectionfont{\centering} % căn giữa section
\setcounter{tocdepth}{1}
\setcounter{tocdepth}{2}
\setcounter{tocdepth}{3}
\setcounter{tocdepth}{4}
\setcounter{tocdepth}{5}
\setcounter{section}{1}
\newtheorem{theorem}{Theorem}[section]
\newtheorem{corollary}{Corollary}[theorem]
\newtheorem{lemma}[theorem]{Lemma}
\setlength\tabcolsep{10pt}
\setlength\arraycolsep{10pt}
\setlength\extrarowheight{10pt}
\renewcommand{\labelenumii}{\arabic{enumi}.\arabic{enumii}}
\renewcommand{\labelenumiii}{\arabic{enumi}.\arabic{enumii}.\arabic{enumiii}}
\renewcommand{\labelenumiv}{\arabic{enumi}.\arabic{enumii}.\arabic{enumiii}.\arabic{enumiv}}
\newtheorem{dl}{Định lý}
\newtheorem{md}{Mệnh đề}
\newtheorem{bd}{Bổ đề}
\newtheorem{hq}{Hệ quả}
\newtheorem{cy}{Chú ý}

\renewcommand{\baselinestretch}{1.2} % Giản dòng 1.2
\setlength{\parskip}{6pt} % Spacing after
\setlength{\parindent}{1cm} % Set khoảng cách thụt đầu dòng mỗi đoạn 

\begin{document}
 \large
\begin{titlepage}
    \begin{tikzpicture}[overlay,remember picture]
    \draw [line width=3pt]
        ($ (current page.north west) + (3.0cm,-2.0cm) $)
        rectangle
        ($ (current page.south east) + (-2.0cm,2.5cm) $);
    \draw [line width=0.5pt]
        ($ (current page.north west) + (3.1cm,-2.1cm) $)
        rectangle
        ($ (current page.south east) + (-2.1cm,2.6cm) $);
    \end{tikzpicture}
    \begin{center}
        \vspace{-15pt} TRƯỜNG ĐẠI HỌC SÀI GÒN\\
        \textbf{\fontsize{16pt}{0pt}\selectfont KHOA TOÁN-ỨNG DỤNG}
        \vspace{0.5cm}
        \begin{figure}[H]
            \centering
            \includegraphics[width=6cm,height=6cm]{imge/logodhsg.png}
        \end{figure}
        \vspace{1.5cm}
        \fontsize{28pt}{0pt}\selectfont BÁO CÁO\\
        \vspace{12pt}
        \textbf{\fontsize{32pt}{0pt}\selectfont NGHIÊN CỨU KHOA HỌC}
        \vspace{1.5cm}
        \end{center}
        \begin{center}
        \hspace{15pt}\textbf{\fontsize{30pt}{0pt}\selectfont Đề tài:}
        \begin{center}
        \textbf{\fontsize{20pt}{0pt}\selectfont TỐI ƯU TUYẾN TÍNH CÓ THAM SỐ} 
        \end{center}
        \vspace{-0.5cm}
        \begin{table}[H]
            \centering
            \begin{tabular}{l l}
        \fontsize{14pt}{0pt}\selectfont Chủ nhiệm đề tài:     & \fontsize{14pt}{0pt}\selectfont NGUYỄN THÀNH NAM - 3122480034 \vspace{6pt}\\     
          & \fontsize{14pt}{0pt}\selectfont LÊ ĐỨC ANH - 3122480001 \vspace{6pt}\\
       
       
        \end{tabular}
        \end{table}
        \vspace{0.5cm}
        \fontsize{14pt}{0pt}\selectfont Thành phố Hồ Chí Minh, tháng 9 năm 2023
        \end{center}
    \end{titlepage}
        \cleardoublepage
       
        \begin{titlepage}
            \begin{tikzpicture}[overlay,remember picture]
            \draw [line width=3pt]
                ($ (current page.north west) + (3.0cm,-2.0cm) $)
                rectangle
                ($ (current page.south east) + (-2.0cm,2.5cm) $);
            \draw [line width=0.5pt]
                ($ (current page.north west) + (3.1cm,-2.1cm) $)
                rectangle
                ($ (current page.south east) + (-2.1cm,2.6cm) $);
            \end{tikzpicture}
            \begin{center}
                \vspace{-15pt} TRƯỜNG ĐẠI HỌC SÀI GÒN\\
                \textbf{\fontsize{16pt}{0pt}\selectfont KHOA TOÁN-ỨNG DỤNG}
                \vspace{0.5cm}
                \begin{figure}[H]
                    \centering
                    \includegraphics[width=6cm,height=6cm]{imge/logodhsg.png}
                \end{figure}
                \vspace{1.5cm}
                \fontsize{24pt}{0pt}\selectfont BÁO CÁO\\
                \vspace{12pt}
                \textbf{\fontsize{32pt}{0pt}\selectfont NGHIÊN CỨU KHOA HỌC}
                \vspace{1cm}
                \end{center}
                \begin{center}
                \hspace{15pt}\textbf{\fontsize{30pt}{0pt}\selectfont Đề tài:}
                \end{center}
                \begin{center}
                    \begin{center}
                \textbf{\fontsize{20pt}{0pt}\selectfont TỐI ƯU TUYẾN TÍNH CÓ THAM SỐ} 
                    \end{center}
                \vspace{1cm}
                \begin{table}[H]
                    \centering
                    \begin{tabular}{l l}
                \fontsize{14pt}{0pt}\selectfont Chủ nhiệm đề tài:     & \fontsize{14pt}{0pt}\selectfont NGUYỄN THÀNH NAM - 3122480034 \vspace{6pt}\\     
                  & \fontsize{14pt}{0pt}\selectfont LÊ ĐỨC ANH - 3122480001 \vspace{6pt}\\
                \fontsize{14pt}{0pt}\selectfont Giảng viên hướng dẫn: & \fontsize{14pt}{0pt}\selectfont PGS.TS TẠ QUANG SƠN \vspace{6pt}\\
                \fontsize{14pt}{0pt}\selectfont Các cán bộ phản biện:
                
               
                \end{tabular}
                \end{table}
                \vspace{-0.5cm}
                \fontsize{14pt}{0pt}\selectfont Thành phố Hồ Chí Minh, tháng 9 năm 2023
                \end{center}
            \end{titlepage}
            \cleardoublepage
\tableofcontents
\section*{\textbf{\fontsize{25}{25}\selectfont Lời nói đầu}}

\begin{center}
\textbf{\fontsize{25}{25}\selectfont Lời cam đoan}
\end{center}
\cleardoublepage
\section*{\uuline{Chương 1}\\
\centering
\vspace{0.5cm}
CƠ SỞ LÝ THUYẾT QUY HOẠCH TUYẾN TÍNH}
\addcontentsline {toc} {section} {Chương 1. Lý thuyết quy hoạch tuyến tính}
\subsection{Bài toán đơn hình}
\subsubsection{Thuật toán  M}
\subsubsection{Thuật toán 2 pha}
\subsection{Bài toán cải biên}
\subsubsection{Thuật toán đơn hình cải biên}
\subsubsection{Thuật toán đơn hình đối ngẫu}
\subsection{Bài toán đối ngẫu}
\subsubsection{Qui tắc cho bài toán đối ngẫu }
\begin{equation*}
    \begin{split}
        M &= \{1,2,3, \ldots , m\} , M_1 = \{ 1,2,3, \ldots , m_1 \} , m_1 \leq m \\
        N &= \{ 1,2,3, \ldots , n\} , N_1 = \{ 1,2,3, \ldots , n_1\} , n_1 \leq n
    \end{split}
\end{equation*}
Trường hợp 1: $(P) \: Min \rightarrow (D) Max$ \\
\begin{center}
    \begin{tabular}{|c|c|c|c|}
        \hline
        & Gốc $(P)$ & $\Rightarrow$& Đối ngẫu $(D)$ \\
        \hline
        1. & $c^Tx \rightarrow min$ && $b^Ty \rightarrow max$ \\
        \hline
        2. &  $\sum_{j=1}^n a_{ij} \geq b_i$ & $i \in M_1$ & $y_i \geq  0 , i \in M_1$ \\
        \hline
        3. & $\sum_{j=1}^n a_{ij} = b_i$ & $i \in M \backslash M_1$ & $y_i$ tự do , $i \in M\backslash M_1$ \\
        \hline
        4. & $x_j \geq 0$ & $j \in N_1$ & $\sum_{i=1}^m a_{ij}y_i \leq c_j , j \in N_1$ \\
        \hline
        5. & $x_j$ có dấu tuỳ ý & $j \in N_1$ & $\sum_{i=1}^m a_{ij} y_i = c_j , j \in N\backslash N_1$ \\
        \hline
        6. & $x_j \leq 0$ & $j \in N_2$ & $\sum_{i=1}^m a_{ij}y_i \geq c_j , j \in N_2$ \\
        \hline
    \end{tabular}
\end{center}
Chú ý: Nếu có $\sum_{j=1}^n \leq b_i$, thì theo nguyên 
tắc đối ngẫu sẽ ứng với biến $y_i \leq 0$. 
Trong thực tế nên chuyển về trường hợp 2 bằng cách
 nhân cả hai vế của bất đẳng thức(bđt) cho $-1$. \\
\vspace{20pt}
\\            
Trường hợp 2: $(P) \: Max \rightarrow (D) Min$ \\
\begin{center}
    \begin{tabular}{|c|c|c|c|}
        \hline
        & Gốc $(P)$ & $\Rightarrow$ & Đối ngẫu $(D)$ \\
        \hline
        1. & $c^Tx \rightarrow max$ && $b^Ty \rightarrow min$ \\
        \hline
        2. &  $\sum_{j=1}^n a_{ij} \leq b_i$ & $i \in M_1$ & $y_i$ tự do, $i \in M_1$ \\
        \hline
        3. & $\sum_{j=1}^n a_{ij} = b_i$ & $i \in M \backslash M_1$ & $y_i$ tự do , $i \in M\backslash M_1$ \\
        \hline
        4. & $x_j \geq 0$ & $j \in N_1$ & $\sum_{i=1}^m a_{ij}y_i \geq c_j , j \in N_1$ \\
        \hline
        5. & $x_j$ có dấu tuỳ ý & $j \in N_1$ & $\sum_{i=1}^m a_{ij} y_i = c_j , j \in N\backslash N_1$ \\
        \hline
        6. & $x_j \leq 0$ & $j \in N_2$ & $\sum_{i=1}^m a_{ij}y_i \leq c_j , j \in N_2$ \\
        \hline
    \end{tabular}
\end{center}
Chú ý: Nếu có $\sum_{j=1}^n a_{ij} \geq b_i$, thì theo nguyên tắc 
đối ngẫu sẽ ứng với biến $y_i \leq 0$. Trong thực tế nên chuyển về
 trường hợp 2 bằng cách nhân cả hai vế của bđt cho $-1$. (Các ràng
  buộc ẩn tự do thường ta không cần ghi và bỏ qua nó).
\subsubsection{Lý thuyết đối ngẫu dạng chuẩn tắc}
Bài toán chuẩn tắc có dạng như sau,
 với $A \in M_{m \times n},b \in R^m, c \in R^n$ \\ 
\begin{equation*}
    \begin{split}
        (P) \quad c^Tx &\rightarrow Min \\
        Ax &\geq b \\
        x &\geq 0 
    \end{split}
\end{equation*}
Với bài toán như trên ta có bài toán đối ngẫu như sau với $y \in R^m$ : \\
\begin{equation*}
    \begin{split}
        (D) \quad b^Ty &\rightarrow Max \\
        A^Ty &\leq c \\
        y &\geq 0
    \end{split}
\end{equation*}
Cặp bài toán trên được gọi là cặp bài toán đối ngẫu đối xứng vì chứa đủ ba phần : Hàm mục tiêu,ràng buộc đẳng thức (bất đẳng thức) và biến không âm. \\
\begin{dl}
    Với mọi phương án $x$ của bài toán $(P)$ và mọi phương án y của bài toán $(P^*)/(D)$, ta có $$g(y) \leq f(x)$$
\end{dl}
\begin{dl}
    \item Nếu bài toán $(P)$ có phương án tối ưu thì bài toán $(P^*)/(D)$ cũng có phương án tối ưu và ngược lại. Đồng thời $V(P)=V(P^*)$.
    \item Nếu hàm mục tiêu của bài toán này không bị chặn thì tập các phương án của bài toán kia là rỗng.
\end{dl}
\begin{hq}
    Nếu các bài toán $(P)$ và $(P^*)$ đều có các phương án khác rỗng thì chúng đều có phương án tối ưu.
\end{hq}
\begin{hq}
    Giả sử $x^*$ và $y^*$ lần lượt là các phương án tối ưu tương ứng của bài toán $(P)$ và $(P^*)$. Ta có:
   \[V(P)=f(x^*)=g(y^*)=V(P^*)\]
\end{hq}
\begin{hq}
    Giả sử $x^*$ và $y^*$ tương ứng là các phương án chấp nhận được của các bài toán $(P)$ và $(P^*)/(D)$. Nếu $f(x^*)=g(y^*)$ thì $x^*$ và $y^*$ lần lượt là phương án tối ưu của $(P)$ và $(P^*)/(D)$.
\end{hq}
\begin{dl}{\textbf{Độ lệch bù trong đối ngẫu đối xứng}}\\
    Cho $x^*$ và $y^*$ tương ứng là phương án của bài toán $(P)$ và $(P^*)$. Phương án $x^*$ và $y^*$ tối ưu khi và chỉ khi:
    \[\left\{\begin{aligned}
        & x_j^* > 0 \Longrightarrow \displaystyle\sum\limits_{i=1}^na_{ij}y_i^*=c_j \;\vee\;\displaystyle\sum\limits_{i=1}^na_{ij}y_i^* < c_j\Longrightarrow x_j^* =0,j=\overline{1,n}\\
        & y_j^* > 0 \Longrightarrow \displaystyle\sum\limits_{i=1}^na_{ij}x_i^*=b_j \;\vee\;\displaystyle\sum\limits_{i=1}^na_{ij}x_i^* < b_j\Longrightarrow y_j^* =0,j=\overline{1,m}
    \end{aligned}\right.\]
\end{dl}
\subsubsection{Lý thuyết đối ngẫu dạng chính tắc}
Bài toán chính tắc có dạng như sau, với $A \in M_{m \times n},b \in R^m, c \in R^n$:
\begin{equation}
    \begin{split}
        (P) \quad c^Tx &\rightarrow Min \\
        Ax&=b \\
        x &\geq 0
    \end{split}
\end{equation}
Với bài toán như trên ta có bài toán đối ngẫu như sau với $y \in R^m$:
\begin{equation}
    \begin{split}
        (D) \quad b^Ty &\rightarrow Max \\
        A^Ty &\leq c
    \end{split}
\end{equation}
"Bài toán đối ngẫu không đối xứng như trên có tất cả cá tính chất như đối ngẫu đối xứng". \\
Định lí về độ lệch bù trong đối ngẫu không đối xứng (hay còn gọi là độ lệch bù yếu).
\begin{dl}
Cặp $(x^*,y^*)$ tướng ứng là nghiệm tối ưu của bài toán $(1)$ và $(2)$ khi và chỉ khi
\[\text{Nếu } x_j^* > 0 \text{ thì } \sum_{i=1}^ma_{ij}y_i^*=c_j \text{ hoặc}\] \\
\[\text{Nếu } \sum_{i=1}^ma_{ij}y_i^* < c_j \text{ thì } x_j^*=0\]
\end{dl}
\subsubsection{Ví dụ minh hoạ}
\hspace{-1cm}
\textbf{1. Cho bài toán quy hoạch tuyến tính (QHTT) như sau:}\\
\vspace{-0.4cm}
\begin{center}
$\left(P\right)$ \quad $f(x)=2x_1-3x_2+4x_3-6x_4\longrightarrow Min$\\
\end{center}\\

\[\left\{\begin{aligned}
x_1+2x-2+3x_3-x_4&=20 \quad(1)\\
-3x_1-x_2+7x_3+7x_4&\leq 32 \quad(2)\\
2x_1+4x_2+x_3+x_4&\geq 18 \quad(3)\\
x_i\geq 0,\forall i=1,2,3
\end{aligned}\right.\]\\
$\Longrightarrow$Ta có bài toán đối ngẫu như sau :$ \big(D\big)\quad g(y)=20y_1+32y_2+18y_3 \longrightarrow Max$ \\
\[\left\{\begin{aligned}
y_1-3y_2+2y_3&\leq 2 \quad(4)\\
2y_1-y_2+4y_3&\leq-3 \quad(5)\\
3y_1+7y_2+y_3&\leq 4 \quad(6)\\
-y_1+7y_2+y_3&=-6 \quad(7)\\
y_2\leq 0,y_3\geq 0
\end{aligned}\right.\]\\
Cặp ràng buộc : $\begin{array}{cr} 
x_1\geq 0\; \&(4) &y_2\leq 0\; \&(2)\\
x_2\geq 0\; \&(5) &y_3\leq 0\; \&(3)\\
\end{array}$\\
\textbf{2.Cho bài toán QHTT như sau:}\\
\vspace{-0.4cm}
\begin{center}
\big(P\big)\quad $f(x)=2x_1+2x_2+x_3+x_4 \longrightarrow Max$\\
\end{center}
\[\left\{\begin{aligned}
5x_1+x_2+x_3+6x_4&=50 \\
-3x_1+x_3+2x_4&\geq16 \\
4x_1+3x_3+x_4&\leq23 \\
x_i\geq 0,i=1,2,3,4 
\end{aligned}\right.\]\\
\begin{enumerate}
\item Viết bài toán đối ngẫu \big(D\big)
\item Cho biết $x^*$=$\left(0,14,6,5\right)$ là nghiệm tối ưu của bài toán
 $\left(P\right)$.Tìm nghiệm tối ưu của bài toán đối ngẫu $\left(D\right)$
\end{enumerate}
\begin{center}
\textbf{Giải}
\end{center}
i.Ta có bài toán đối ngẫu như sau:\\
\vspace{-0.5cm}
\begin{center}
$\left(D\right)$: $g(y)=50y_1+16y_2+23y_3 \longrightarrow  Min$\\
\end{center}

\[\left\{\begin{aligned}
5y_1-3y_2+4y_3&\geq2 \\
y_1&\geq2 \\
y_1+y_2+3y_3&\geq1 \\
6y_1+2y_2+y_3&\geq1 \\
y_2\leq 0,y_3\geq 0
\end{aligned}\right.\]\\

Áp dụng định lí độ lệch bù của bài toán đối ngẫu đối xứng,với \textbf{PATƯ} của $\left(P\right)$ là 
$\left(0,14,6,5\right)$ có $x_2\geq 0,x_3\geq 0,x_4\geq 0$ ta suy ra :$\left\{\begin{aligned} 
y_1&=2 \\
y_1+y_2+3y_3&=1\quad;\\
6y_1+2y_2+y_3&=1 
\end{aligned}\right.$\\

$\Longleftrightarrow \quad y^*=\left(2,\frac{-32}{5},\frac{9}{5}\right)$

Vậy nghiệm tối ưu của bài toán đối ngẫu $\left(D\right)$ là $y^*=\left(2,\dfrac{-32}{5},\dfrac{9}{5}\right)$

\subsection{Tài liệu tham khảo}


\cleardoublepage
\section*{\uline{Chương 2}\\
\centering
\vspace{0.5cm}
 TỐI ƯU TUYẾN TÍNH CHỨA THAM SỐ}
 \addcontentsline {toc} {section} {Chương 2. Tối ưu tuyến tính chứa tham số}
 Trong thực tế,ở một số mô hình Bài toán QHTT, 
 các hệ số ban đầu như
  $a_{ij},b_i,c_j,i=\overline{1,m},j=\overline{1.n}$,
  có thể không xác định một cách chính xác mà có thể dao động,
   biến đổi phụ thuộc vào một hay nhiều tham số như thời gian,
\ bão lụt,\ tắc đường,\ nguyên liệu v.v\ldots Đồng thời sẽ phải giải từng bài 
toán quy hoạch  tuyến tính ứng với từng giá trị khác nhau của tham số đấy thì
chi phí tính toán phát sinh rất lớn.Do đó việc tìm phương án tối ưu
 cho bài toán QHTT sẽ bị mất đi ý nghĩa kinh tế của nó.\\
 \qquad Để khắc phục tình trạng này,







\subsection{Phương pháp giải Bài toán QHTT chứa tham số ở hàm mục tiêu}
\subsubsection{Cơ sở lý thuyết và thuật toán}
Ta xét Bài toán quy hoạch sau đây:\;

\textit{Tìm giá trị của $x_1,x_2.\ldots,x_n$ làm cực tiểu hàm
 mục tiêu}\;

 \begin{equation}
  f(x)=\displaystyle\sum\limits_{j=1}^n(c_j+d_jt)x_j
   \longrightarrow Min
 \end{equation}\;





\subsection{Phương pháp giải Bài toán QHTT chứa tham số vế phải ràng buộc}



           


\end{document}